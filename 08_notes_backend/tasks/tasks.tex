\documentclass[a4paper,12pt]{article}
\usepackage[utf8]{inputenc}
\usepackage[ngerman]{babel}
\usepackage{amsmath}
\usepackage{parskip}

\setlength{\parindent}{0pt}

\title{Modul 294 - Aufgabenblatt 08}
\author{von Lukas Meier}
\date{Unterricht vom 13.02.2026}

\begin{document}

\maketitle

\section{Vorbereitung}

Legen Sie in Ihrer JavaScript-Datei zwei Konstanten an:
\begin{itemize}
    \item \texttt{BASE\_URL=http://notes.iso.lmeier.ch}
    \item \texttt{NOTES\_ENDPOINT=/api/collections/notes/records}
\end{itemize}

Diese beinhalten die URL des Backendes sowie den Endpoint auf welchem Notizen geladen und erstellt werden. 

\section{GET-Request: Notizen laden}

Erstellen Sie eine Funktion \texttt{loadNotes()}, die alle Notizen vom Backend lädt.

Anforderungen:
\begin{itemize}
    \item \texttt{fetch} mit Methode GET verwenden
    \item \texttt{response.ok} prüfen
    \item Die JSON-Daten als Array zurückgeben
\end{itemize}

\section{Notizen im DOM anzeigen}

Erstellen Sie eine Funktion \texttt{renderNotes(notes)}.

Diese Funktion soll:
\begin{itemize}
    \item ein vorhandenes \texttt{ul}- oder \texttt{div}-Element leeren
    \item jede Notiz im DOM darstellen
    \item mindestens Titel und Inhalt anzeigen
\end{itemize}

\section{App-Start}

Rufen Sie beim Laden der Seite zuerst \texttt{loadNotes()} auf und übergeben Sie das Resultat an \texttt{renderNotes()}.

Prüfen Sie, ob alle vorhandenen Notizen korrekt angezeigt werden.

\section{Formular vorbereiten}

Erstellen oder ergänzen Sie ein Formular mit:
\begin{itemize}
    \item Eingabefeld für den Titel
    \item Textfeld für den Inhalt
    \item Submit-Button
\end{itemize}

Fangen Sie den Submit mit einem Event-Listener ab und verhindern Sie das Standardverhalten.

\section{POST-Request: Notiz erstellen}

Erstellen Sie eine Funktion \texttt{createNote(noteData)}.

Anforderungen:
\begin{itemize}
    \item \texttt{fetch} mit Methode \texttt{POST}
    \item Header \texttt{Content-Type: application/json}
    \item Body mit \texttt{JSON.stringify(noteData)}
    \item \texttt{response.ok} prüfen
\end{itemize}

\section{Formular mit Backend verbinden}

Beim Absenden des Formulars:
\begin{itemize}
    \item Daten aus den Inputs auslesen
    \item an \texttt{createNote()} senden
    \item Formular zurücksetzen
\end{itemize}

\section{Liste aktualisieren}

Nach erfolgreichem POST-Request soll die Anzeige aktualisiert werden.

Wählen Sie eine Variante:
\begin{itemize}
    \item \textbf{Variante A:} Nach dem Erstellen erneut \texttt{loadNotes()} aufrufen und neu rendern
    \item \textbf{Variante B:} Die neue Notiz direkt lokal zur Liste hinzufügen und nur ein Element nachrendern
\end{itemize}

\section{Fehlerbehandlung}

Erweitern Sie \texttt{loadNotes()} und \texttt{createNote()} um \texttt{try / catch}.

Im Fehlerfall:
\begin{itemize}
    \item Fehlermeldung in der Konsole ausgeben
    \item sichtbare Meldung im UI anzeigen (z.\,B. ``Backend nicht erreichbar'')
\end{itemize}

\section{Debugging-Aufgabe}

Erzeugen Sie absichtlich einen Fehler, indem Sie den Endpoint falsch schreiben.

Dokumentieren Sie kurz:
\begin{itemize}
    \item Welche Meldung in der Konsole erscheint
    \item Wie Sie den Fehler erkannt und behoben haben
\end{itemize}

\section{Abschluss}

Wenn alles funktioniert:
\begin{itemize}
    \item Notizen werden beim Start geladen
    \item Neue Notizen werden gespeichert
    \item Liste aktualisiert sich sichtbar
\end{itemize}

Erstellen Sie anschliessend einen Commit mit einer klaren Nachricht, z.\,B.
\texttt{feat: connect notes app to backend with GET and POST}.

\section{Optional}

Erweitern Sie die App um:
\begin{itemize}
    \item Validierung (leere Eingaben verhindern)
    \item Ladezustand (``Lade...'')
\end{itemize}

\end{document}
