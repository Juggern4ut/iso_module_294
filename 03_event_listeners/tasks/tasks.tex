\documentclass[a4paper,12pt]{article}
\usepackage[utf8]{inputenc}
\usepackage[ngerman]{babel}
\usepackage{amsmath}
\usepackage{parskip}

\setlength{\parindent}{0pt}

\title{Modul 294 - Aufgabenblatt 03}
\author{von Lukas Meier}
\date{Unterricht vom 09.12.2025}

\begin{document}

\maketitle

\section{Vorbereitung}

Erstellen Sie einen neuen Ordner und speichern Sie die Datei \texttt{index.html} aus dem Netzwerkordner darin. Öffnen Sie die Datei anschliessend im Code-Editor Ihrer Wahl und fügen Sie ein \texttt{script}-Tag ein, falls dieses noch nicht existiert.

\section{Erste Funktion}

Schreiben Sie eine Funktion \texttt{begruessen(name)}, welche eine personalisierte Begrüssung in der Konsole ausgibt. Rufen Sie die Funktion zweimal mit unterschiedlichen Namen auf.

\section{Return-Werte}

Erstellen Sie eine Funktion \texttt{addieren(a, b)}, welche die beiden übergebenen Werte addiert und \texttt{return} zur Rückgabe des Ergebnisses nutzt.

Geben Sie das Resultat anschliessend in der Konsole aus.  
Erstellen Sie zusätzlich eine weitere Funktion \texttt{laenge(text)}, welche die Länge des übergebenen Strings zurückgibt.

\section{Anonyme Funktion und Arrow Function}

Wandeln Sie folgende normale Funktion zuerst in eine anonyme Funktion und danach in eine Arrow Function um. Testen Sie alle drei Varianten im Browser:

\begin{verbatim}
function multiplizieren(x, y) {
    return x * y;
}
\end{verbatim}

\section{IIFE}

Schreiben Sie eine selbstausführende Funktion (IIFE), welche beim Laden der Seite in der Konsole den Text \glqq Programm gestartet\grqq{} ausgibt.

\section{EventListener hinzufügen}

Ergänzen Sie folgendes HTML in der \texttt{index.html}, falls es noch nicht vorhanden ist:

\begin{verbatim}
<button id="btn">Klick mich</button>
<p id="ausgabe"></p>
\end{verbatim}

Fügen Sie im JavaScript einen EventListener hinzu, welcher bei jedem Klick auf den Button im Element \texttt{ausgabe} den Text \glqq Der Button wurde gedrückt\grqq{} setzt. Nutzen Sie dafür eine normale Funktion.

\section{EventListener mit Arrow Function}

Ersetzen Sie nun die Funktion aus der vorherigen Aufgabe durch eine Arrow Function und prüfen Sie, ob alles wie zuvor funktioniert.

\section{EventListener entfernen}

Schreiben Sie eine benannte Funktion \texttt{melden()}, die den Text \glqq Klick erkannt\grqq{} in der Konsole ausgibt.  
Fügen Sie diese Funktion als \texttt{click}-EventListener auf den Button hinzu und entfernen Sie sie nach dem ersten Klick sofort wieder.

(Hinweis: \texttt{removeEventListener} funktioniert nur mit benannten Funktionen.)

\section{Mehrere Events}

Fügen Sie eine zweite Interaktion hinzu:

\begin{itemize}
    \item Ein \texttt{mouseover}-Event auf dem Button soll den Text im \texttt{ausgabe}-Element auf \glqq Maus darüber\grqq{} setzen.
    \item Ein \texttt{mouseout}-Event soll den Text wieder löschen.
\end{itemize}

\section{Unterschiedliche Funktionstypen anwenden}

Erstellen Sie zu jedem Event aus Abschnitt 9 einen anderen Funktionstyp:

\begin{itemize}
    \item \texttt{click}: normale benannte Funktion
    \item \texttt{mouseover}: anonyme Funktion
    \item \texttt{mouseout}: Arrow Function
\end{itemize}

\section{Event-Toggle}

Erstellen Sie einen zweiten Button mit dem Text \texttt{Toggle}.  
Wenn dieser Button gedrückt wird, soll der EventListener auf dem Eingabefeld entweder aktiviert oder deaktiviert werden – je nach aktuellem Zustand.

(Hinweis: Nutzen Sie \texttt{classList.contains} oder eine eigene Statusvariable.)

\section{Mini-Projekt: Dynamische Eingabe}

Erweitern Sie das HTML um folgendes Element:

\begin{verbatim}
<input id="eingabe" type="text" placeholder="Geben Sie etwas ein...">
\end{verbatim}

Erstellen Sie nun eine Funktion \texttt{textAktualisieren()}, welche bei jeder Eingabe den aktuellen Inhalt des Feldes in das \texttt{ausgabe}-Element schreibt.

Binden Sie diese Funktion über einen \texttt{input}-EventListener ein.

\section{Mini-Projekt: Interaktive Farbwahlen}

Erweitern Sie das HTML-Dokument um die folgenden Elemente:

\begin{verbatim}
<div id="farbBox" style="width:150px; height:150px; border:1px solid #000;">
</div>

<label for="rot">Rot:</label>
<input id="rot" type="range" min="0" max="255" value="125">

<label for="gruen">Grün:</label>
<input id="gruen" type="range" min="0" max="255" value="125">

<label for="blau">Blau:</label>
<input id="blau" type="range" min="0" max="255" value="125">

<button id="reset">Zurücksetzen</button>
\end{verbatim}

Erstellen Sie nun eine kleine Anwendung, welche es ermöglicht, die Farbe der Box dynamisch anzupassen. Gehen Sie dabei wie folgt vor:

\begin{enumerate}
    \item Schreiben Sie eine Funktion \texttt{farbeAktualisieren()}, welche die aktuellen Werte der drei Regler ausliest und daraus eine gültige CSS-\texttt{rgb}-Farbe zusammensetzt.
    \item Sorgen Sie dafür, dass die Funktion bei jeder Veränderung eines der Regler ausgeführt wird. Nutzen Sie hierzu den \texttt{input}-EventListener.
    \item Setzen Sie die Hintergrundfarbe der \texttt{farbBox} dynamisch anhand der berechneten Werte.
    \item Schreiben Sie ausserdem eine Funktion \texttt{zuruecksetzen()}, welche alle Regler wieder auf den Ausgangswert \texttt{125} setzt und die Box wieder auf eine neutrale Farbe stellt.
    \item Binden Sie diese Funktion an den Button \texttt{reset}.
    \item Nutzen Sie bei der Umsetzung der Funktionen unterschiedliche Funktionstypen (normale Funktion, anonyme Funktion, Arrow Function), wie in den vorherigen Aufgaben behandelt.
    \item Erweitern Sie das Projekt so, dass die erzeugte Farbe zusätzlich in einem Absatz unterhalb der Box als Text ausgegeben wird, z.\,B.\ \glqq Aktuelle Farbe: rgb(125, 200, 50)\grqq{}.
\end{enumerate}

\end{document}
