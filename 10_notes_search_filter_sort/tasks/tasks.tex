\documentclass[a4paper,12pt]{article}
\usepackage[utf8]{inputenc}
\usepackage[ngerman]{babel}
\usepackage{amsmath}
\usepackage{parskip}

\setlength{\parindent}{0pt}

\title{Modul 294 - Aufgabenblatt 10}
\author{von Lukas Meier}
\date{Unterricht vom 27.02.2026}

\begin{document}

\maketitle

\section{Vorbereitung}

Nutzen Sie den aktuellen Projektstand mit funktionierendem CRUD (GET, POST, PATCH/PUT, DELETE).

Legen Sie eine zentrale Datenvariable an, z.\,B. \texttt{allNotes}, die alle vom Backend geladenen Notizen enthält.

\section{UI-Elemente für Suche, Filter und Sortierung}

Ergänzen Sie im HTML mindestens folgende Elemente:
\begin{itemize}
    \item Suchfeld (Textinput)
    \item Filter (Select oder Buttons, z.\,B. alle / offen / erledigt)
    \item Sortierung (Select, z.\,B. neueste / älteste / Titel A-Z)
\end{itemize}

\section{UI-State definieren}

Erstellen Sie ein State-Objekt, das die aktuelle Auswahl speichert, z.\,B.:
\begin{itemize}
    \item \texttt{searchTerm}
    \item \texttt{status}
    \item \texttt{sortBy}
\end{itemize}

\section{Suche implementieren}

Erstellen Sie eine Funktion \texttt{applySearch(notes, term)}.

Anforderungen:
\begin{itemize}
    \item Suche über Titel und Inhalt
    \item Gross-/Kleinschreibung ignorieren
    \item Leerer Suchbegriff liefert alle Notizen zurück
\end{itemize}

\section{Filter implementieren}

Erstellen Sie eine Funktion \texttt{applyFilter(notes, status)}.

Nutzen Sie den Status Ihrer Datenstruktur (z.\,B. \texttt{done}, \texttt{completed} oder \texttt{archived}).

Der Filter soll mindestens 3 Zustände unterstützen:
\begin{itemize}
    \item alle
    \item offen
    \item erledigt
\end{itemize}

\section{Sortierung implementieren}

Erstellen Sie eine Funktion \texttt{applySort(notes, sortBy)}.

Unterstützen Sie mindestens:
\begin{itemize}
    \item neueste zuerst (Datum absteigend)
    \item älteste zuerst (Datum aufsteigend)
    \item Titel alphabetisch
\end{itemize}

Wichtig: Verändern Sie nicht direkt das Original-Array, sondern arbeiten Sie mit einer Kopie.

\section{Pipeline kombinieren}

Erstellen Sie eine Funktion \texttt{getVisibleNotes()} mit folgendem Ablauf:
\begin{itemize}
    \item Suche anwenden
    \item Filter anwenden
    \item Sortierung anwenden
\end{itemize}

Das Ergebnis soll an \texttt{renderNotes(...)} übergeben werden.

\section{Events anbinden}

Binden Sie Events an alle UI-Controls:
\begin{itemize}
    \item \texttt{input} beim Suchfeld
    \item \texttt{change} bei Filter und Sortierung
\end{itemize}

Bei jeder Änderung:
\begin{itemize}
    \item State aktualisieren
    \item Ansicht neu rendern
\end{itemize}

\section{Integration mit bestehendem CRUD}

Nach Create, Update und Delete muss die Liste weiterhin korrekt angezeigt werden.

Vorgehen:
\begin{itemize}
    \item Daten neu laden oder lokal aktualisieren
    \item danach \texttt{getVisibleNotes()} und \texttt{renderNotes()} ausführen
\end{itemize}

\section{Testfälle}

Testen Sie mindestens folgende Fälle:
\begin{itemize}
    \item Suche nach Begriffen, die in mehreren Notizen vorkommen
    \item Suche nach Begriff, der nicht existiert
    \item Kombination aus Suche + Filter
    \item Kombination aus Suche + Filter + Sortierung
\end{itemize}

Dokumentieren Sie kurz, welche Kombinationen Sie getestet haben.

\section{Optional: Usability}

Implementieren Sie eine oder mehrere Erweiterungen:
\begin{itemize}
    \item ``Keine Treffer''-Hinweis
    \item ``Filter zurücksetzen''-Button
    \item Debounce für die Suche (z.\,B. 200ms)
\end{itemize}

\section{Abschluss}

Die Lektion ist abgeschlossen, wenn:
\begin{itemize}
    \item Suche, Filter und Sortierung einzeln funktionieren
    \item alle drei Funktionen kombiniert stabil laufen
    \item CRUD weiterhin korrekt funktioniert
\end{itemize}

Erstellen Sie danach einen Commit, z.\,B.
\texttt{feat: add note search, filter and sorting controls}.

\end{document}
