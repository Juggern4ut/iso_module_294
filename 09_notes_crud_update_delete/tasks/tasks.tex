\documentclass[a4paper,12pt]{article}
\usepackage[utf8]{inputenc}
\usepackage[ngerman]{babel}
\usepackage{amsmath}
\usepackage{parskip}

\setlength{\parindent}{0pt}

\title{Modul 294 - Aufgabenblatt 09}
\author{von Lukas Meier}
\date{Unterricht vom 20.02.2026}

\begin{document}

\maketitle

\section{Vorbereitung}

Nutzen Sie den Stand aus Lektion 08 (GET + POST funktionieren bereits).

Stellen Sie sicher, dass jede Notiz eine eindeutige \texttt{id} besitzt und im Frontend verfügbar ist.

\section{Notizen mit ID rendern}

Erweitern Sie Ihre \texttt{renderNotes(notes)}-Funktion so, dass pro Notiz die ID im HTML verfügbar ist.

Beispiel:
\begin{itemize}
    \item als \texttt{data-id} am Container
    \item oder als \texttt{data-delete-id} / \texttt{data-edit-id} an Buttons
\end{itemize}

\section{Delete-Button ergänzen}

Fügen Sie bei jeder dargestellten Notiz einen \texttt{Loeschen}-Button hinzu.

Beim Klick soll die korrekte Notiz-ID bestimmt werden.

\section{DELETE-Request implementieren}

Erstellen Sie eine Funktion \texttt{deleteNote(id)}.

Anforderungen:
\begin{itemize}
    \item Endpoint: \texttt{/notes/:id}
    \item Methode: \texttt{DELETE}
    \item Status mit \texttt{response.ok} prüfen
\end{itemize}

\section{Delete-Flow abschliessen}

Verbinden Sie den Delete-Button mit \texttt{deleteNote(id)}.

Nach erfolgreichem Loeschen:
\begin{itemize}
    \item Liste neu laden und rendern
    \item oder geloeschtes Element direkt aus dem DOM entfernen
\end{itemize}

\section{Edit-Button ergänzen}

Fügen Sie zusätzlich einen \texttt{Bearbeiten}-Button pro Notiz ein.

Beim Klick sollen Titel und Inhalt in ein Edit-Formular übernommen werden.

Speichern Sie die aktuell bearbeitete ID in einer Variable (z.\,B. \texttt{currentEditId}).

\section{Update-Request implementieren}

Erstellen Sie eine Funktion \texttt{updateNote(id, noteData)}.

Anforderungen:
\begin{itemize}
    \item Endpoint: \texttt{/notes/:id}
    \item Methode: \texttt{PATCH} (oder \texttt{PUT}, falls Ihr Backend das so erwartet)
    \item Header: \texttt{Content-Type: application/json}
    \item Body: \texttt{JSON.stringify(noteData)}
    \item Status prüfen
\end{itemize}

\section{Edit-Formular absenden}

Beim Submit des Edit-Formulars:
\begin{itemize}
    \item Standardverhalten verhindern
    \item neue Werte auslesen
    \item \texttt{updateNote(currentEditId, noteData)} aufrufen
    \item Formular zurücksetzen
    \item Liste aktualisieren
\end{itemize}

\section{Validierung ergänzen}

Verhindern Sie bei Create und Update leere Eingaben.

Wenn ein Feld leer ist:
\begin{itemize}
    \item Request nicht absenden
    \item kurze Fehlermeldung im UI anzeigen
\end{itemize}

\section{Fehlerbehandlung und Debugging}

Bauen Sie in \texttt{deleteNote()} und \texttt{updateNote()} eine \texttt{try/catch}-Behandlung ein.

Testen Sie zwei Fehlerfälle:
\begin{itemize}
    \item falscher Endpoint
    \item Update auf nicht existierende ID
\end{itemize}

Dokumentieren Sie kurz, welche Meldungen im Browser erscheinen.

\section{Abschluss-Check}

Die App ist fertig, wenn Folgendes funktioniert:
\begin{itemize}
    \item Notizen laden
    \item Notizen erstellen
    \item Notizen bearbeiten
    \item Notizen loeschen
\end{itemize}

Erstellen Sie anschliessend einen Commit, z.\,B.
\texttt{feat: add note update and delete flows}.

\section{Optional}

Erweitern Sie den Bearbeitungsflow um eines der folgenden Features:
\begin{itemize}
    \item Inline-Editing direkt im Listeneintrag
    \item Bestaetigungsdialog vor dem Loeschen
    \item Undo-Funktion (nur im Frontend, zeitlich begrenzt)
\end{itemize}

\end{document}
