\documentclass[a4paper,12pt]{article}
\usepackage[utf8]{inputenc}
\usepackage[ngerman]{babel}
\usepackage{amsmath}
\usepackage{parskip}

\setlength{\parindent}{0pt}

\title{Modul 294 - Aufgabenblatt 11}
\author{von Lukas Meier}
\date{Unterricht vom 06.03.2026}

\begin{document}

\maketitle

\section{Vorbereitung}

Nutzen Sie den aktuellen Stand Ihrer Notizen-App (CRUD funktioniert).

Heute wird eine einfache Authentifizierung ergänzt:
\begin{itemize}
    \item Benutzername
    \item Passwort
    \item Bearer-Token
\end{itemize}

\section{Login-Formular erstellen}

Erstellen Sie ein Formular mit folgenden Feldern:
\begin{itemize}
    \item \texttt{username}
    \item \texttt{password} (Typ \texttt{password})
    \item Login-Button
\end{itemize}

Fangen Sie den Submit per JavaScript ab und verhindern Sie das Standardverhalten.

\section{Login-Request implementieren}

Erstellen Sie eine Funktion \texttt{login(username, password)}.

Anforderungen:
\begin{itemize}
    \item Endpoint: \texttt{POST /auth/login} (oder Ihr Backend-Pendant)
    \item Header: \texttt{Content-Type: application/json}
    \item Body enthält Benutzername und Passwort als JSON
    \item Status via \texttt{response.ok} prüfen
\end{itemize}

\section{Token aus der Antwort verarbeiten}

Extrahieren Sie das Token aus der Login-Antwort und speichern Sie es in \texttt{localStorage}.

Implementieren Sie Hilfsfunktionen:
\begin{itemize}
    \item \texttt{saveToken(token)}
    \item \texttt{getToken()}
    \item \texttt{clearToken()}
\end{itemize}

\section{Auth-Request-Helper bauen}

Erstellen Sie eine Funktion \texttt{fetchWithAuth(url, options)}.

Diese Funktion soll:
\begin{itemize}
    \item das gespeicherte Token lesen
    \item den Header \texttt{Authorization: Bearer <token>} setzen
    \item den Request mit \texttt{fetch} senden
\end{itemize}

\section{Bestehende Notiz-Requests umstellen}

Stellen Sie Ihre bestehenden Requests auf den Auth-Helper um:
\begin{itemize}
    \item \texttt{GET /notes}
    \item \texttt{POST /notes}
    \item \texttt{PATCH/PUT /notes/:id}
    \item \texttt{DELETE /notes/:id}
\end{itemize}

Alle geschuetzten Endpoints sollen nur noch mit Bearer-Token angesprochen werden.

\section{401-Handling implementieren}

Behandeln Sie den Fall \texttt{response.status === 401} zentral:
\begin{itemize}
    \item Token loeschen
    \item Notiz-Bereich ausblenden
    \item Login-Bereich wieder einblenden
    \item Meldung anzeigen (z.\,B. ``Bitte erneut einloggen'')
\end{itemize}

\section{Logout-Funktion}

Erstellen Sie einen Logout-Button.

Beim Klick:
\begin{itemize}
    \item Token loeschen
    \item App-Ansicht sperren/zuruecksetzen
    \item Login-Form erneut zeigen
\end{itemize}

\section{Testszenarien}

Testen Sie mindestens:
\begin{itemize}
    \item Login mit korrekten Daten
    \item Login mit falschem Passwort
    \item Zugriff auf \texttt{/notes} ohne Token
    \item Zugriff auf \texttt{/notes} mit Token
    \item Verhalten nach Logout
\end{itemize}

Notieren Sie kurz pro Testfall das Ergebnis.

\section{Fehlerbehandlung}

Verwenden Sie \texttt{try/catch} bei Login und geschuetzten Requests.

Geben Sie Fehler gleichzeitig aus:
\begin{itemize}
    \item in der Konsole (Debug)
    \item im UI (verstaendliche Rueckmeldung)
\end{itemize}

\section{Abschluss}

Die Lektion ist abgeschlossen, wenn:
\begin{itemize}
    \item Login und Logout funktionieren
    \item Token korrekt gesetzt und verwendet wird
    \item geschuetzte Requests mit und ohne Token erwartetes Verhalten zeigen
\end{itemize}

Erstellen Sie anschliessend einen Commit, z.\,B.
\texttt{feat: add basic bearer token authentication flow}.

\section{Optional}

Falls Zeit bleibt:
\begin{itemize}
    \item Token-Ablauf simulieren und erneuten Login erzwingen
    \item Kleine Session-Anzeige im UI (``Eingeloggt als ...'')
    \item Login bei Seitenstart automatisch pruefen (wenn Token vorhanden)
\end{itemize}

\end{document}
