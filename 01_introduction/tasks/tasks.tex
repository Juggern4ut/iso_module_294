\documentclass{article}

\usepackage[utf8]{inputenc}
\usepackage{amsmath}

\title{Modul 294 - Aufgabenblatt 01}
\author{Lukas Meier}
\date{\today}

\begin{document}

\maketitle

\pagebreak

\section*{Aufgaben}
Um die soeben behandelte Theorie zu festigen und um einen ersten Eindruck von der Anwendungsweise von JavaScript zu erhalten. Sollten sie die folgenden Aufgaben der Reihe nach abschliessen.

\subsection*{Einrichten der Entwicklungsumgebung}

Da JavaScript nativ in jedem modernen Browser ausgeführt werden kann, ist keine Installation einer Runtime notwendig. 
Um den Code optimal zu bearbeiten, sollten Sie jedoch eine IDE oder einen simplen Codeeditor installiert haben. 
Beliebte Lösungen hierfür sind: \textbf{Visual Studio Code}, \textbf{Sublime Text} oder \textbf{NetBeans}. (Oder, für die fanatischten unter euch NeoVim.)

\subsection*{}

\end{document}
