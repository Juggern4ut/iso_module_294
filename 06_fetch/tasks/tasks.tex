\documentclass[a4paper,12pt]{article}
\usepackage[utf8]{inputenc}
\usepackage[ngerman]{babel}
\usepackage{amsmath}
\usepackage{parskip}

\setlength{\parindent}{0pt}

\title{Modul 294 - Aufgabenblatt 06}
\author{von Lukas Meier}
\date{Unterricht vom 06.01.2026}

\begin{document}

\maketitle

\section{Vorbereitung}

Erstellen Sie einen neuen Ordner und speichern Sie die Datei \texttt{index.html} darin.  
Öffnen Sie die Datei im Browser sowie im Code-Editor Ihrer Wahl.

Stellen Sie sicher, dass ein \texttt{script}-Tag im HTML vorhanden ist.

\section{Wiederholung: Einfacher GET-Request}

Nutzen Sie die \texttt{fetch}-Methode, um Daten vom folgenden Endpoint zu laden:

\texttt{https://jsonplaceholder.typicode.com/posts}

Geben Sie die geladenen Daten vollständig in der Konsole aus.  
Verwenden Sie hierfür die \texttt{then}-Syntax.

\section{GET mit async / await}

Schreiben Sie die Fetch-Anfrage aus der vorherigen Aufgabe so um, dass sie mit \texttt{async / await} umgesetzt ist.

Lagern Sie den Code in eine Funktion \texttt{loadPosts()} aus und rufen Sie diese auf.

\section{Response und Statuscode}

Geben Sie zusätzlich den HTTP-Statuscode der Response in der Konsole aus.

Prüfen Sie mithilfe von \texttt{response.ok}, ob die Anfrage erfolgreich war.  
Falls nicht, soll ein Fehler ausgelöst werden.

\section{Fehlerbehandlung}

Erweitern Sie die Funktion aus der vorherigen Aufgabe um eine Fehlerbehandlung mit \texttt{try / catch}.

Geben Sie im Fehlerfall eine aussagekräftige Meldung in der Konsole aus.

\section{Fetch mit Options-Objekt}

Erstellen Sie einen Fetch-Request, bei welchem Sie explizit ein Options-Objekt übergeben.

Setzen Sie dabei mindestens folgende Option:
\begin{itemize}
    \item \texttt{method}
\end{itemize}

Beobachten Sie, wie sich der Request verhält.

\section{Header mitsenden}

Erweitern Sie den Fetch-Request so, dass ein Header mitgesendet wird.

Setzen Sie den Header \texttt{Content-Type} auf \texttt{application/json}.

Geben Sie die vollständige Response in der Konsole aus.

\section{POST-Request vorbereiten}

Erstellen Sie ein JavaScript Objekt mit folgenden Properties:
\begin{itemize}
    \item title
    \item body
    \item userId
\end{itemize}

\section{POST-Request mit fetch}

Senden Sie das Objekt aus der vorherigen Aufgabe mithilfe eines POST-Requests an folgenden Endpoint:

\texttt{https://jsonplaceholder.typicode.com/posts}

Beachten Sie dabei:
\begin{itemize}
    \item Setzen Sie die Methode auf \texttt{POST}
    \item Senden Sie den passenden Header
    \item Wandeln Sie den Body korrekt in JSON um
\end{itemize}

Geben Sie die Antwort des Servers in der Konsole aus.

\section{POST mit async / await}

Schreiben Sie den POST-Request aus der vorherigen Aufgabe so um, dass er mit \texttt{async / await} umgesetzt ist.

Lagern Sie den Code in eine Funktion \texttt{createPost()} aus.

\section{Antwort auswerten}

Geben Sie nach erfolgreichem POST-Request folgende Informationen aus:
\begin{itemize}
    \item ID des erstellten Eintrags
    \item Titel des Eintrags
\end{itemize}

Beschreiben Sie kurz, was Ihnen an der Antwort auffällt.

\end{document}

