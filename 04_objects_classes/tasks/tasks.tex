\documentclass[a4paper,12pt]{article}
\usepackage[utf8]{inputenc}
\usepackage[ngerman]{babel}
\usepackage{amsmath}
\usepackage{parskip}

\setlength{\parindent}{0pt}

\title{Modul 294 - Aufgabenblatt 04}
\author{von Lukas Meier}
\date{Unterricht vom 16.12.2025}

\begin{document}

\maketitle

\section{Vorbereitung}

Erstellen Sie einen neuen Ordner und speichern Sie die Datei \texttt{index.html} aus dem Netzwerkordner darin. Öffnen Sie die Datei anschliessend im Browser und im Code-Editor Ihrer Wahl.

Stellen Sie sicher, dass ein \texttt{script}-Tag im Dokument vorhanden ist.

\section{Erstes Objekt}

Erstellen Sie ein JavaScript Objekt mit dem Namen \texttt{person}.  
Das Objekt soll folgende Properties besitzen:

\begin{itemize}
    \item \texttt{name}
    \item \texttt{alter}
    \item \texttt{wohnort}
\end{itemize}

Geben Sie anschliessend alle Properties einzeln in der Browser-Konsole aus.

\section{Methoden im Objekt}

Erweitern Sie das Objekt \texttt{person} um eine Methode \texttt{greet()}, welche in der Konsole eine persönliche Begrüssung ausgibt.

Nutzen Sie innerhalb der Methode das Keyword \texttt{this}, um auf die Properties des Objekts zuzugreifen.

\section{Objekte verändern}

Passen Sie das Objekt \texttt{person} wie folgt an:

\begin{itemize}
    \item Ändern Sie den Wert von \texttt{alter}
    \item Fügen Sie eine neue Property \texttt{beruf} hinzu
\end{itemize}

Geben Sie das aktualisierte Objekt vollständig in der Konsole aus.

\section{Mehrere Objekte}

Erstellen Sie ein zweites Objekt \texttt{person2} mit denselben Properties, aber anderen Werten.

Vergleichen Sie die beiden Objekte in der Konsole und beschreiben Sie kurz, was Ihnen auffällt.

\section{Einführung Klassen}

Erstellen Sie eine Klasse \texttt{Person}.  
Die Klasse soll einen \texttt{constructor} besitzen, welcher folgende Werte entgegennimmt:

\begin{itemize}
    \item name
    \item alter
    \item wohnort
\end{itemize}

Speichern Sie diese Werte als Properties der Klasse.

\section{Methoden in Klassen}

Erweitern Sie die Klasse \texttt{Person} um eine Methode \texttt{vorstellen()}, welche in der Konsole einen vollständigen Vorstellungssatz ausgibt.

Beispiel: \glqq Hallo, ich heisse Max, bin 20 Jahre alt und wohne in Bern.\grqq

\section{Instanzen erstellen}

Erstellen Sie mithilfe des \texttt{new}-Keywords mindestens zwei Instanzen der Klasse \texttt{Person}.

Rufen Sie anschliessend auf beiden Instanzen die Methode \texttt{vorstellen()} auf.

\section{Klassen und Methoden erweitern}

Erweitern Sie die Klasse \texttt{Person} um eine weitere Methode \texttt{hatGeburtstag()}, welche das Alter der Person um 1 erhöht.

Testen Sie die Methode an einer bestehenden Instanz und geben Sie das neue Alter aus.

\section{Objekte im DOM verwenden}

Erstellen Sie im HTML einen Button mit einer passenden ID sowie ein leeres \texttt{p}-Element.

Wenn der Button geklickt wird, soll eine zuvor erstellte \texttt{Person}-Instanz ihre Vorstellung im \texttt{p}-Element anzeigen.

Nutzen Sie dafür einen EventListener und greifen Sie auf die Methode der Klasse zu.

\section{Mehrere Instanzen verwalten}

Erstellen Sie ein Array, welches mehrere \texttt{Person}-Instanzen enthält.

Schreiben Sie eine Funktion, welche alle Personen aus dem Array der Reihe nach in der Konsole ausgibt oder im DOM darstellt.

\section{Mini-Projekt: Benutzerverwaltung}

Erstellen Sie eine kleine Benutzerverwaltung:

\begin{itemize}
    \item Erstellen Sie eine Klasse \texttt{User} mit den Properties \texttt{username} und \texttt{email}
    \item Fügen Sie eine Methode \texttt{anzeigen()} hinzu, welche die Benutzerdaten formatiert zurückgibt
    \item Erstellen Sie mehrere User-Instanzen
    \item Speichern Sie diese in einem Array
    \item Geben Sie alle Benutzer beim Laden der Seite im Browser aus
\end{itemize}

Achten Sie auf saubere Struktur und verständliche Variablennamen.

\section{Mini-Projekt 2: Interaktive Benutzerliste}

Erstellen Sie eine interaktive Benutzerverwaltung im Browser:

\begin{itemize}
    \item Verwenden Sie wieder die Klasse \texttt{User} mit den Properties \texttt{username} und \texttt{email} sowie der Methode \texttt{anzeigen()}.
    \item Erstellen Sie ein Formular, über das neue Benutzer angelegt werden können (Eingabefelder für \texttt{username} und \texttt{email} und ein "Hinzufügen"-Button).
    \item Speichern Sie die Benutzer in einem Array.
    \item Zeigen Sie alle Benutzer dynamisch in einer Liste auf der Webseite an.
    \item Fügen Sie für jeden Benutzer einen "Löschen"-Button hinzu, mit dem der Benutzer aus der Liste entfernt werden kann.
    \item Achten Sie auf eine übersichtliche Darstellung und verständliche Variablennamen.
\end{itemize}

\end{document}
