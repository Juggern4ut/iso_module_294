\documentclass[a4paper,12pt]{article}
\usepackage[utf8]{inputenc}
\usepackage[ngerman]{babel}
\usepackage{amsmath}
\usepackage{parskip}

\setlength{\parindent}{0pt}

\title{Modul 294 - Aufgabenblatt 05}
\author{von Lukas Meier}
\date{Unterricht vom 23.12.2025}

\begin{document}

\maketitle

\section{Vorbereitung}

Erstellen Sie einen neuen Ordner und speichern Sie die Datei \texttt{index.html} aus dem Netzwerkordner darin.  
Öffnen Sie die Datei im Browser sowie in einem Code-Editor Ihrer Wahl und stellen Sie sicher, dass ein \texttt{script}-Tag vorhanden ist.

\section{Synchroner und asynchroner Code}

Schreiben Sie zwei \texttt{console.log}-Ausgaben direkt untereinander.  
Erstellen Sie anschliessend einen \texttt{setTimeout}, welcher nach 2 Sekunden eine weitere Meldung in der Konsole ausgibt. Ein Timeout kann in JavaScript mit der folgenden Syntax realisiert werden: \texttt{setTimeout(function, millisekunden)}. Damit wird die funktion im ersten Parameter nach der anzahl Millisekunden des zweiten Parameters ausgeführt.

Beobachten und beschreiben Sie die Reihenfolge der Ausgaben.

\section{Erstes Promise}

Erstellen Sie ein Promise, welches nach 1 Sekunde erfolgreich aufgelöst wird und den Text \glqq Promise erfüllt\grqq{} zurückgibt.

Geben Sie das Resultat mithilfe von \texttt{then} in der Konsole aus.

\section{Promise mit Fehler}

Erweitern Sie das Promise aus der vorherigen Aufgabe so, dass es alternativ abgelehnt wird.

Fangen Sie den Fehler mit \texttt{catch} ab und geben Sie eine sinnvolle Fehlermeldung in der Konsole aus.

\section{Asynchrone Funktion}

Schreiben Sie eine Funktion \texttt{ladeDaten()}, welche mit dem Keyword \texttt{async} definiert ist und einen Text zurückgibt.

Rufen Sie die Funktion auf und geben Sie den Rückgabewert in der Konsole aus.  
Beobachten Sie, welchen Datentyp der Rückgabewert hat.

\section{await verwenden}

Erstellen Sie innerhalb einer \texttt{async}-Funktion ein Promise, welches nach 2 Sekunden den Wert \texttt{42} zurückgibt.

Nutzen Sie \texttt{await}, um den Wert in einer Variable zu speichern, und geben Sie diese anschliessend aus.

\section{Fehlerbehandlung mit try / catch}

Erweitern Sie die Funktion aus der vorherigen Aufgabe so, dass das Promise auch fehlschlagen kann.

Behandeln Sie diesen Fall mit einem \texttt{try / catch}-Block und geben Sie eine Fehlermeldung in der Konsole aus.

\section{Fetch – erste Anfrage}

Nutzen Sie die \texttt{fetch}-Methode, um Daten von folgendem Endpoint zu laden:

\texttt{https://jsonplaceholder.typicode.com/todos/}

Geben Sie die geladenen Daten vollständig in der Konsole aus.  
Verwenden Sie hierfür \texttt{then}.

\section{Fetch mit async / await}

Schreiben Sie die Fetch-Anfrage aus der vorherigen Aufgabe um, sodass sie mithilfe von \texttt{async / await} umgesetzt ist.

Lagern Sie den Code in eine Funktion \texttt{loadTodos()} aus und rufen Sie diese beim Laden der Seite auf.

\section{ToDos im DOM darstellen}

Erstellen Sie im HTML ein leeres \texttt{ul}-Element.

Stellen Sie nun alle geladenen ToDos als \texttt{li}-Elemente dar.  
Jedes ToDo soll mindestens folgende Informationen anzeigen:

\begin{itemize}
    \item Titel
    \item Status (completed / not completed)
\end{itemize}

Kennzeichnen Sie erledigte ToDos visuell (z.\,B.\ durch Text oder CSS-Klasse).

\section{User zu ToDos laden}

Laden Sie zusätzlich die Benutzerdaten vom folgenden Endpoint:

\texttt{https://jsonplaceholder.typicode.com/users/}

Die ToDos enthalten eine \texttt{userId}, welche einem Benutzer entspricht.

Ordnen Sie jedem ToDo den passenden Benutzernamen zu und zeigen Sie diesen ebenfalls im \texttt{li}-Element an.

\section{Kombinierte Darstellung}

Erweitern Sie die Darstellung der ToDos so, dass pro Eintrag folgende Informationen sichtbar sind:

\begin{itemize}
    \item Titel des ToDos
    \item Name des Benutzers
    \item Status des ToDos
\end{itemize}

Achten Sie darauf, dass die Daten sauber strukturiert und übersichtlich dargestellt sind.

\section{Optional: Filter}

Erweitern Sie die Anwendung optional um eine Filterfunktion:

\begin{itemize}
    \item Ein Button zeigt nur erledigte ToDos
    \item Ein weiterer Button zeigt nur nicht erledigte ToDos
\end{itemize}

Nutzen Sie hierfür bereits geladene Daten und führen Sie keine erneuten Fetch-Anfragen aus.

\end{document}
